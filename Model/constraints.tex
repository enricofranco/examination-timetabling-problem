\section*{Variables and constraints}
\subsection*{Input data}
$ \vec{Me} $ is a $ n \times n $ matrix, where $ n = |E| $ is the total number of exams, containing the number of students in conflit among two exams. More precisely, $ Me_{ij} $ contains the number of students in conflit among the exam $i$ and the exam $j$, where $ i, j \in \mathbb{N}, 1 \le i \le |E|, 1 \le j \le |E| $.
\newline
$ \vec{Me} $ is built directly from the input data.
\newline
$ |S| $ is the total number of enrolled students.

\subsection*{Variables}
$ \vec{p} $ is a vector containing the penalty value associated to exams. More precisely, $ p_{k} $ contains the penalty weight associated to exams which are $k$ timeslots far, where $ k \in \mathbb{N}, 1 \le k \le 5 $. Hence $p_{k} = 2^{k}$.
\newline
$ \vec{y} $ is a 3d matrix containing the information about conflicts at a certain distance. More precisely, 
\[
y_{ijk} = 
\begin{cases}
1	&	\text{if exams $ i $ and $ j $ are $k$ timeslot far}	\\
0	&	\text{otherwise}
\end{cases}
\]

\subsection*{Output data}
$ \vec{x} $ is a vector containing the number of the timeslot assigned to an exam. More precisely, $ x_{i} $ contains the number of the timeslot assigned to the exam $i$.

\subsection*{Constraints}
$ Me_{ij} > 0 $ means that there are some students that can not substain the same exam in the same timeslot. Hence, it is needed to force $ x_{i} \ne x_{j} $.
In order to express this condition, it is necessary to introduce two new variables
\[
l_{ij} = 
\begin{cases}
1	&	Me_{ij} > 0	\\
0	&	\text{otherwise}
\end{cases}
\]
\[
q_{ij} = 
\begin{cases}
1	&	x_{i} \ne x_{j}	\\
0	&	\text{otherwise}
\end{cases}
\]
and to generate the following constraints:
\begin{gather*}
Me_{ij} \le M l_{ij} \qquad Me_{ij} \ge m l_{ij} \\
x_{i} - x_{j} \ge m q_{ij} \qquad x_{i} - x_{j} \le M q_{ij} \\
l_{ij} \le q_{ij}
\end{gather*}
\newline
It is needed to activate the proper $ y_{ijk} $ in order to represent the correct couple of exams in conflict at a certain distance.
\newline
$ x_{i} - x_{j} = k $ means that exam $i$ and exam $j$ are $k$ slots far. Hence, it is needed to force $ y_{ijk} = 1 $ where $ k \in \mathbb{N}, 1 \le k \le 5 $.
In order to express this condition, it is necessary to introduce two different blocks of constraints.

The first one forces $ y_{ijk} = 0 $ when $ x_{i} - x_{j} \ne k $ in order to "facilitate" the minimization of the objective function.
\begin{gather*}
k - (x_{i} - x_{j}) \le M (1-y_{ijk}) \\
(x_{i} - x_{j}) - k \le M (1-y_{ijk})
\end{gather*}

The second one forces $ y_{ijk} = 1 $ when $ x_{i} - x_{j} = k $.
In order to express this constrains it is needed to introduce two more boolean variables, $ y_{ijk, I} $ and $ y_{ijk, II} $.
\begin{gather}
\label{eqn:1}
(x_{i} - x_{j}) - k - \epsilon \ge (m - \epsilon) y_{ijk, I} \\
\label{eqn:2}
(x_{i} - x_{j}) - k + \epsilon \ge (M + \epsilon) y_{ijk, II} \\
\label{eqn:3}
y_{ijk, I} + y_{ijk, II} - 1 \le y_{ijk}
\end{gather}
With the relation \ref{eqn:1} the following constraint is expressed:
\[
y_{ijk, I} = 
\begin{cases}
1	&	\text{if $ (x_{i} - x_{j}) \ge k $}	\\
0	&	\text{otherwise}
\end{cases}
\]
Analagously, with the relation \ref{eqn:2} the following constraint is expressed:
\[
y_{ijk, II} = 
\begin{cases}
1	&	\text{if $ (x_{i} - x_{j}) \le k $}	\\
0	&	\text{otherwise}
\end{cases}
\]
Finally, the relation \ref{eqn:3} forces $ y_{ijk} = 1 $ only if $ y_{ijk, I} = 1 $ and $ y_{ijk, II} = 1 $, that represents the situation in which $ x_{i} - x_{j} = k $.