\section*{Solution improvement}
%In order to improve the solution, we exploited the following approach:
The idea behind our algorithm is that, in the previous stage, we found some groups of exams which make no conflict among each other. In order to improve the solution and not to lose the feasibility, we swap those ``independent'' exams trying to distance the most problematic ones, i.e. the ones with a lot of booked students.
So, we exploited the following algorithm:

Loop until the timeout expires:
\begin{enumerate}
\item Check if a mutation is needed and mutate consequently
\item For each exam:
\begin{enumerate}
\item Search the timeslots where the exam does not conflict and where the move is not tabu
\item If there exists at least one available timeslot, for each one of them:
\begin{enumerate}
\item Move the exam to the new timeslot
\item Evaluate the objective function
\end{enumerate}
\item Swap to the best position and mark the move as tabu
\item Save the new solution
\end{enumerate}
\end{enumerate}
We mutate the solution after $|E|$, i.e. number of exams, iteration in the main loop. This is an empiric number related to the size of the problem which provides good results.
\paragraph*{Mutation}
\begin{enumerate}
\item Choose a random timeslot $t_1$
\item Save the previous solution temporarily
\item Find the best timeslot where to swap for the exams in $t_1$:
\begin{enumerate}
\item Swap all exams in $t_1$ with the ones in other timeslots
\item Evaluate the objective function for each swap
\item Save the best position
\end{enumerate}
\item I choose the best timeslot among the available ones, then swap the exams and mark the move as tabu
\item Save the new solution and evaluate its objective function
\item If it is an improving one, save it. Otherwise restore the previous saved one
\end{enumerate}