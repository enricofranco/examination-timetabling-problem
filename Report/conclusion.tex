\section*{Conclusions}
In our first attempts to find a set of feasible solutions, we tried simpler approaches, such as a simple taboo list, a steepest descent using first improvement strategy and a quite sophisticated genetic algorithm performing both swaps and random mutations but they were not effective. Because of this difficulties, we adopted a mixed approach. In fact, in both parts we use some greedy concepts, a taboo list and the idea of a periodic mutation to explore a different region in the solution space.

Our algorithm is very fast in the search of feasible solution. In most cases, it provides a set of feasible solutions in less than a second for all instances, except \emph{instance06}, because of its huge number of exams. The assignation loop cycles for some hundreds of thousands of iterations because it continuously adds and removes exams in different timeslots. In this case, the algorithm is able to find a set of feasible solutions in 8-10 seconds at most.

The optimization phase shows the same problem of the previous phase. In fact, this stage is quite fast for all instances and it produces good results which do not improve much for longer running times while, for \emph{instance06}, we noticed that it continues to improve. The reason for this behavior should be, in this case too, the huge number of exams: a single swap between two different timeslots requires to change a larger number of exams on average. Obviously, this slows down the execution of a single optimization cycle and it also affects the overall performance.

\paragraph*{Benchmark}
In order to evaluate the algorithm performance and gather some statistics we run 60 times each instance using a 60 seconds timeout. We represent with $s\,\%$ the standard deviation weighted on the average value.
\begin{center}
\small
\begin{tabular}{lrrrcc}
\toprule
Instance	&	Benchmark	&	Avg value	&	Max value	&	Avg Gap \%	&	$s\,\%$	\\
\midrule
instance01	&	157.033		&	157.289		&	157.444		&	0.16		&	0.05	\\
instance02	&	34.709		&	37.634		&	38.236		&	8.43		&	0.76	\\
instance03	&	32.627		&	35.039		&	36.359		&	7.39		&	1.35	\\
instance04	&	7.717		&	8.654		&	8.930		&	12.14		&	1.32	\\
instance05	&	12.901		&	14.786		&	15.293		&	14.61		&	1.40	\\
instance06	&	3.045		&	3.790		&	3.916		&	24.48		&	1.55	\\
instance07	&	10.050		&	10.805		&	11.009		&	7.51		&	0.88	\\
\bottomrule
\end{tabular}
\end{center}
It is possible to notice the quality of the algorithm, which in most cases produces good results with Gap < 15\% and appreciate its stability. In fact, on every instance it provides the same result, in terms of objective function, with an error < 2\%.